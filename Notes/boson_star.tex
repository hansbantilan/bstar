\documentclass[prl,twocolumn,superscriptaddress]{revtex4-1}
\pdfoutput=1
\usepackage[letterpaper,top=1.45cm,bottom=1.65cm,left=2cm,right=2cm]{geometry}
\usepackage{amssymb,amsmath,amsthm,graphicx}
\usepackage{subfigure}
\usepackage{graphics, color}
\usepackage{latexsym}
\usepackage{bm}
\usepackage{epsfig}
\usepackage{multirow,tabularx}
\usepackage[colorlinks=true,linktocpage=true,linkcolor=blue,citecolor=blue]{hyperref}
\usepackage[abs]{overpic}

\usepackage{fancyhdr}
\fancyhf{}
\cfoot{\thepage}
\renewcommand{\headrulewidth}{0pt}
\pagestyle{fancy}
\fancypagestyle{plain}

\setlength{\parskip}{0cm}

\newcommand{\PRLsection}[1]{\emph{#1.---}}

\newcommand{\ToDo}[1]{\textbf{\textsf !#1!}}
\newcommand{\TF}{\mathrm{TF}}  %TF
\newcolumntype{Y}{>{\centering\arraybackslash}X}

\newtheorem{theorem}{Theorem}[section]
\newtheorem{example}[theorem]{Example}
\newtheorem{definition}[theorem]{Definition}
\newtheorem{exercise}[theorem]{Exercise}
\newtheorem{proposition}[theorem]{Proposition}
\newtheorem{note}[theorem]{Note}
\newtheorem{lemma}[theorem]{Lemma}
\newtheorem{corollary}[theorem]{Corollary}
\newtheorem{unusedproblem}[theorem]{unused Problem}
\newtheorem{question}[theorem]{Question}
\newtheorem{project}[theorem]{Project}
\newtheorem{problem}[theorem]{Problem}
\newtheorem{conjecture}[theorem]{Conjecture}
\newtheorem{remark}[theorem]{Remark}


%%%%%%%%%%%%%%%%%%%%%%%%
% Document
%%%%%%%%%%%%%%%%%%%%%%%%
\begin{document}

\title{Boson stars as holographic heavy ions}

\author{}

\maketitle

%-------------------------------------------------------
% Preliminaries
%-------------------------------------------------------
\PRLsection{Preliminaries}
Near the AdS boundary $r\rightarrow\infty$, a complex scalar $\varphi$ with potential $V(\varphi)=\mu^2 |\varphi|^2 /2$ takes on the asymptotic form\cite{Henneaux:2006hk,Marolf:2006nd}
\begin{equation}
\varphi \rightarrow \frac{A}{r^{\lambda_-}} + \frac{B}{r^{\lambda_+}},
\end{equation}
where 
\begin{equation}
\lambda_{\pm} = \frac{D-1}{2}+\sqrt{\left( \frac{D-1}{2} \right)^2 + \mu^2 L^2},
\end{equation}
and the AdS curvature length scale is $L$. 
To reproduce the asymptotics shown in (26) of~\cite{Astefanesei:2003qy}, which takes the form $\varphi \sim \hat{\varphi}_0 r^c$ for the power $c=-\left[ \frac{D-1}{2} + \sqrt{\left( \frac{D-1}{2} \right)^2 - \frac{(D-1)(D-2)}{2\Lambda}} \right]$, first note the rescaling $\Lambda \rightarrow \Lambda/\mu^2$ used in that work, and remember the relation between the AdS length scale $L$ and the cosmological constant $\Lambda=-\frac{(D-1)(D-2)}{2 L^2}$.
The rescaling then gives the expected $c\rightarrow -\lambda_+$.

The important take-away message: all numerical values of the cosmological constant quoted in~\cite{Astefanesei:2003qy} are in fact $\nu\equiv\Lambda/\mu^2$. 
We are free to fix the AdS length scale to be $L=1$, which is the scale with respect to which all other lengths are measured. 
All numerical values of $\nu\equiv\Lambda/\mu^2$ then implies a mass parameter value $\mu^2=-(D-1)(D-2)/(2\nu)$ e.g. $\nu=-0.1$ in $D=5$ gives $\mu^2=60$.

%-------------------------------------------------------
% Bibliography
%-------------------------------------------------------
\bibliography{boson_star}
\bibliographystyle{apsrev4-1}

\end{document}
