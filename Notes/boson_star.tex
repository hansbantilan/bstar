\documentclass[prl,twocolumn,superscriptaddress]{revtex4-1}
\pdfoutput=1
\usepackage[letterpaper,top=1.45cm,bottom=1.65cm,left=2cm,right=2cm]{geometry}
\usepackage{amssymb,amsmath,amsthm,graphicx}
\usepackage{subfigure}
\usepackage{graphics, color}
\usepackage{latexsym}
\usepackage{bm}
\usepackage{epsfig}
\usepackage{multirow,tabularx}
\usepackage[colorlinks=true,linktocpage=true,linkcolor=blue,citecolor=blue]{hyperref}
\usepackage[abs]{overpic}

\usepackage{fancyhdr}
\fancyhf{}
\cfoot{\thepage}
\renewcommand{\headrulewidth}{0pt}
\pagestyle{fancy}
\fancypagestyle{plain}

\setlength{\parskip}{0cm}

\newcommand{\PRLsection}[1]{\emph{#1.---}}

\newcommand{\ToDo}[1]{\textbf{\textsf !#1!}}
\newcommand{\TF}{\mathrm{TF}}  %TF
\newcolumntype{Y}{>{\centering\arraybackslash}X}

\newtheorem{theorem}{Theorem}[section]
\newtheorem{example}[theorem]{Example}
\newtheorem{definition}[theorem]{Definition}
\newtheorem{exercise}[theorem]{Exercise}
\newtheorem{proposition}[theorem]{Proposition}
\newtheorem{note}[theorem]{Note}
\newtheorem{lemma}[theorem]{Lemma}
\newtheorem{corollary}[theorem]{Corollary}
\newtheorem{unusedproblem}[theorem]{unused Problem}
\newtheorem{question}[theorem]{Question}
\newtheorem{project}[theorem]{Project}
\newtheorem{problem}[theorem]{Problem}
\newtheorem{conjecture}[theorem]{Conjecture}
\newtheorem{remark}[theorem]{Remark}


%%%%%%%%%%%%%%%%%%%%%%%%
% Document
%%%%%%%%%%%%%%%%%%%%%%%%
\begin{document}

\title{Boson stars as holographic heavy ions}

\author{}

\maketitle

%-------------------------------------------------------
% Preliminaries
%-------------------------------------------------------
\PRLsection{Preliminaries}
Near the AdS boundary $r\rightarrow\infty$, a complex scalar $\varphi$ with potential $V(\varphi)=\mu^2 |\varphi|^2 /2$ takes on the asymptotic form\cite{Henneaux:2006hk,Marolf:2006nd}
\begin{equation}
\varphi \rightarrow \frac{A}{r^{\lambda_-}} + \frac{B}{r^{\lambda_+}},
\end{equation}
where 
\begin{equation}
\lambda_{\pm} = \frac{D-1}{2}+\sqrt{\left( \frac{D-1}{2} \right)^2 + \mu^2 L^2},
\end{equation}
and the AdS curvature length scale is $L$. 
To reproduce the asymptotics shown in (26) of~\cite{Astefanesei:2003qy}, which takes the form $\varphi \sim \hat{\varphi}_0 r^c$ for the power $c=-\left[ \frac{D-1}{2} + \sqrt{\left( \frac{D-1}{2} \right)^2 - \frac{(D-1)(D-2)}{2\Lambda}} \right]$, first note the rescaling $\Lambda \rightarrow \Lambda/\mu^2$ used in that work, and remember the relation between the AdS length scale $L$ and the cosmological constant $\Lambda=-\frac{(D-1)(D-2)}{2 L^2}$.
The rescaling then gives the expected $c\rightarrow -\lambda_+$.

The important take-away message: all numerical values of the cosmological constant quoted in~\cite{Astefanesei:2003qy} are in fact $\nu\equiv\Lambda/\mu^2$. 
We are free to fix the AdS length scale to be $L=1$, which is the scale with respect to which all other lengths are measured. 
All numerical values of $\nu\equiv\Lambda/\mu^2$ then implies a mass parameter value $\mu^2=-(D-1)(D-2)/(2\nu)$ e.g. $\nu=-0.1$ in $D=5$ gives $\mu^2=60$.

\PRLsection{Transformation to Cartesian Coordinates}
The metric of global AdS$_5$ is
\begin{eqnarray}\label{eqn:originalmetric}
\hat{g}_{\mu\nu} dx^\mu dx^\nu = - f(r) dt^2 + \frac{1}{f(r)} dr^2 +r^2 d\Omega_{(3)}^2,
\end{eqnarray}
where $d\Omega_{(3)}^2=r^2 \left( d\chi^2 + \sin^2\chi \left(d\theta^2 + \sin^2\theta d\phi^2\right) \right)$ is the metric of the three-sphere, and we have defined $f(r)=1+r^2/L^2$ for convenience. 
Here, $L$ is the AdS radius of curvature, related to the cosmological constant in five dimensions by $\Lambda_5 = -(D-1)(D-2)/(2 L^2) = -6/L^2$. 
It is useful to introduce a compactified radial coordinate $\rho$. We choose
\begin{equation}\label{eqn:r_def}
r=\frac{2\rho}{1-\rho^2/\ell^2},
\end{equation}
where $\ell$ is an arbitrary compactification scale, independent of the AdS length scale $L$, such that the AdS boundary is reached when $\rho=\ell$. 
In all of the following, we set $\ell=1$, though note that this scale is implicitly present since $\rho$ has dimensions of length. 
The metric \eqref{eqn:originalmetric} in these compactified coordinates becomes\footnote{The Jacobian of the transformation is
$\frac{\partial r}{\partial \rho} = \frac{2(1+\rho^2)}{(1-\rho^2)^2}$.
}
\begin{eqnarray}
\hat{g}
&=& \frac{1}{(1-\rho^2)^2} \left[ -\hat{f}(\rho) dt^2 + \frac{4(1+\rho^2)^2}{\hat{f}(\rho)} d\rho^2 + 4\rho^2 d\Omega_{(3)}^2) \right], 
\end{eqnarray}
where $\hat{f}(\rho) = (1-\rho^2)^2+4\rho^2$.

We now introduce Cartesian coordinates $x=\rho\cos\chi$, $y=\rho\sin\chi$ in terms of which the compactified coordinates in the previous section are $\rho=\sqrt{x^2+y^2}$, $\chi=\arccos\left( x/\sqrt{x^2+y^2} \right)$.
The metric components in these Cartesian coordinates has the following form in terms of the metric components in polar coordinates\footnote{The elements of the Jacobian of the transformation are
$\frac{\partial \rho}{\partial x} = \frac{x}{\rho}$,
$\frac{\partial \rho}{\partial y} = \frac{y}{\rho}$,
$\frac{\partial \chi}{\partial x} = -\frac{y}{\rho^2}$,
$\frac{\partial \chi}{\partial y} = \frac{x}{\rho^2}$.
}

\begin{eqnarray}\label{eqn:cartesianmetric_components}
g_{xx} &=& \frac{1}{\rho^2} \left( x^2 g_{\rho\rho} + \frac{y^2}{\rho^2} g_{\chi\chi} \right) \nonumber \\
g_{xy} &=& \frac{xy}{\rho^2} \left( g_{\rho\rho} - \frac{1}{\rho^2} g_{\chi\chi} \right) \nonumber \\
g_{yy} &=& \frac{1}{\rho^2} \left( y^2 g_{\rho\rho} + \frac{x^2}{\rho^2} g_{\chi\chi} \right).
\end{eqnarray}

Applying this transformation to the metric~\eqref{eqn:originalmetric} of pure AdS, one obtains
\begin{eqnarray}\label{eqn:ads_cartesianmetric_components}
\hat{g}_{tt} &=& \frac{1}{(1-\rho^2)^2} \left[ -\hat{f}(\rho) \right] \nonumber \\
\hat{g}_{xx} &=& \frac{4}{\rho^2(1-\rho^2)^2} \left[ \frac{(1+\rho^2)^2}{\hat{f}(\rho)} x^2 + y^2 \right] \nonumber \\
\hat{g}_{xy} &=& \frac{4xy}{\rho^2(1-\rho^2)^2} \left[ \frac{(1+\rho^2)^2}{\hat{f}(\rho)} - 1 \right] \nonumber \\
\hat{g}_{yy} &=& \frac{4}{\rho^2(1-\rho^2)^2} \left[ \frac{(1+\rho^2)^2}{\hat{f}(\rho)} y^2 + x^2 \right] \nonumber \\
\end{eqnarray}

Using (8) in~\cite{Astefanesei:2003qy} which in five spacetime dimensions gives components $g_{tt}=-\left( 1+r^2/L^2-2m/r^2\right) e^{-2\delta}$, $g_{rr}=\left( 1+r^2/L^2-2m/r^2\right)^{-1}$, and $g_{\chi\chi}=r^2$.
Transforming to a compactified radial coordinate, the metric components read $g_{tt}=-\left( \frac{\hat{f}(\rho)}{(1-\rho^2)^2} - \frac{(1-\rho^2)^2}{2\rho^2} m \right) e^{-2\delta}$, $g_{\rho\rho}=\frac{4(1+\rho^2)^2}{(1-\rho^2)^2} \frac{1}{\hat{f}(\rho) - \frac{(1-\rho^2)^2}{2\rho^2} m}$, and $g_{\chi\chi}=\frac{4\rho^2}{(1-\rho^2)^2}$.
Using~\eqref{eqn:cartesianmetric_components}, we can now write down the form of (8) in~\cite{Astefanesei:2003qy} in Cartesian coordinates
\begin{eqnarray}\label{eqn:full_cartesianmetric_components}
g_{xx} &=& \frac{4}{\rho^2(1-\rho^2)^2} \left( x^2 \frac{(1+\rho^2)^2}{\hat{f}(\rho) - \frac{(1-\rho^2)^2}{2\rho^2} m} + y^2 \right) \nonumber \\
g_{xy} &=& \frac{4xy}{\rho^2(1-\rho^2)^2} \left( \frac{(1+\rho^2)^2}{\hat{f}(\rho) - \frac{(1-\rho^2)^2}{2\rho^2} m} - 1 \right) \nonumber \\
g_{yy} &=& \frac{4}{\rho^2(1-\rho^2)^2} \left( y^2 \frac{(1+\rho^2)^2}{\hat{f}(\rho) - \frac{(1-\rho^2)^2}{2\rho^2} m} + x^2 \right).
\end{eqnarray}

Finally, our evolved variables are $\bar{g}_{\mu\nu}$ defined in terms of the full metric $g_{\mu\nu}=\hat{g}_{\mu\nu}+(1-\rho^2)\bar{g}_{\mu\nu}$.
We thus subtract~\eqref{eqn:ads_cartesianmetric_components} from~\eqref{eqn:full_cartesianmetric_components} and divide by $1-\rho^2$, to obtain
\begin{eqnarray}\label{eqn:barred_cartesianmetric_components}
\bar{g}_{tt} &=& \frac{\hat{f}(\rho)}{(1-\rho^2)^3} (1-e^{-2\delta}) + \frac{(1-\rho^2)}{2\rho^2} m e^{-2\delta} \nonumber \\
\bar{g}_{xx} &=& \frac{4x^2}{\rho^2(1-\rho^2)} \frac{(1+\rho^2)^2 m}{\hat{f}(\rho) \left( 2\rho^2\hat{f}(\rho)-(1-\rho^2)^2m \right)} \nonumber \\
\bar{g}_{xy} &=& \frac{4xy}{\rho^2(1-\rho^2)} \frac{(1+\rho^2)^2 m}{\hat{f}(\rho) \left( 2\rho^2\hat{f}(\rho)-(1-\rho^2)^2m \right)} \nonumber \\
\bar{g}_{yy} &=& \frac{4y^2}{\rho^2(1-\rho^2)} \frac{(1+\rho^2)^2 m}{\hat{f}(\rho) \left( 2\rho^2\hat{f}(\rho)-(1-\rho^2)^2m \right)}.
\end{eqnarray}

%-------------------------------------------------------
% Bibliography
%-------------------------------------------------------
\bibliography{boson_star}
\bibliographystyle{apsrev4-1}

\end{document}
